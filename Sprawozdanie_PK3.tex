
\documentclass[10pt, a4paper]{article}


% % % Ustawienia strony
\usepackage[margin=0.5in]{geometry}

% % % Język polski
\usepackage[utf8]{inputenc}                                      
\usepackage[T1]{fontenc}                            
\usepackage[polish]{babel}  

% % % Środowisko 
\usepackage{verbatim}


% % % Nagłówek pliku
\newcommand{\KNDate}[1]{#1, \today}
\newcommand{\KNPrzedmiot}[1]{\textbf{#1}}
\newcommand{\KNTemat}[1]{Temat: #1}
\newcommand{\KNSemestr}[1]{Semestr: #1}
\newcommand{\KNGrupa}[1]{Grupa: #1}
\newcommand{\KNAutor}[1]{Autor: #1}
\newcommand{\KNProwadzacy}[1]{Prowadzący: #1}

\newcommand{\KNNaglowek}[8]
{
	\newpage
	\begin{center}
		{
			% % % Brak numeracji strony
			\thispagestyle{empty}	
			\raggedleft \KNDate{#1}
					
			\vspace{5cm}	
			\centering
			\Huge \KNPrzedmiot{#2}\\
			\vspace{1cm}	
			\LARGE \KNTemat{#3} \\
			
			\vspace{15cm}
			\normalsize
			\raggedleft
			\KNAutor{#5}\\
			\KNSemestr{#6}\\
			\KNGrupa{#7}\\
			\vspace{1cm}
			\KNProwadzacy{#4}\\

		
			\newpage	
		} 
	\end{center}
}


\begin{document}
	
\KNNaglowek{Gliwice}
           {Programowanie Komputerów}
           {Baza płyt z oprogramowaniem instalacyjnym}
           {dr inż. Karolina Nurzyńska}
           {Radosław Serba}
           {trzeci}
           {II}
           

\section{Treść zadania} 
Zadaniem programu jest zarządzanie zbiorem plików z oprogramowaniem instalacyjnym. Zarządzanie obejmuje: przeszukiwanie zasobów, dodawanie nowych pozycji, modyfikacje oraz usuwanie istniejących rekordów.
\section{Analiza rozwiązania}
Głównym zadaniem programu jest przechowywanie informacji o przechowywanych plikach oprogramowania instalacyjnego w postaci wide stringów. Przy pierwszym otwarciu programu ten poinformuje nas o tym oraz o utworzeniu pliku konfiguracyjnego przechowującego repozytorium.\\

Następnie program wyświetli użytkownikowi wszystkie możliwe opcje do wykonania począwszy od zmiany lokalizacji repozytorium po dodawanie, usuwanie, edytowanie i wyszukiwanie plików wedle wskazanych kryteriów. Użytkownik porusza się w programie za pomocą uproszczonego menu głównego, przy czym aby się po menu poruszać należy naciskać przyciski cyfr odpowiadjących funkcjom, które użytkownik chce wykorzystać.\\

Do wykonania programu zostały wykorzystane klasy \textbf{Repository} do obsługi zapisu pliku konfiguracyjnego przechowującego repozytorium, \textbf{Menu} do wyświetlania interfejsu użytkownika oraz do zapewnienia dostępu do wszystkich funkcjonalności programu, \textbf{RepoFile} do definicji plików instalacyjnych w postaci obiektów, \textbf{RepoManager} do obsługi listy wspomnianych plików instalacyjnych wraz ze wszystkimi operacjami wspomnianymi w treści zadania oraz \textbf{ValueChecker} jako klasa uzupełniająca do uniknięcia błędów wynikających z naturalnego działania niektórych funkcji w języku C++ dla wcześniej wspominanych klas.
\section{Specyfikacja zewnętrzna}
Program należy uruchomić poprzez kliknięcie dwukrotnie lewym przyciskiem myszy na ikonę w Eksploratorze Windows lub poprzez linię poleceń (cmd.exe) bez podania argumentów, w następujący sposób:\\
\texttt{<FileFinder.exe>}\\
\\
Po uruchomieniu programu, w przypadku jego pierwszego uruchomienia wyświetli komunikat:\\
\texttt{This is the first run of FileFinder. The configuration file will be created in:}\\
\texttt{C:\textbackslash Users\textbackslash <nazwa użytkownika>\textbackslash AppData\textbackslash Roaming\textbackslash FileFinder.cfg}\\
\\
Następnie po naciśnięciu dowolnego klawisza zostaniemy przekierowani do menu głównego programu:\\
\texttt{WARNING: repository location is not set\\
\\
Welcome in Repofinder! Choose desired operation by typing a number:\\
|\\
|---(1) Set new repository location\\
|---(2) Show all files in repository\\
|---(3) Find specified Repo file\\
|---(4) Add Repo file\\
|---(5) Delete Repo file\\
|---(6) Edit Repo file\\
|---(7) Save changes into repository\\
|---(8) Help\\
|---(9) About the project\\
|---(0) Exit}\\
\\
Wybierając opcję (1) jesteśmy przenoszeni do podmenu pozwalającego na ustawienie nowego miejsca na repozytorium plików poprzez podanie odpowiedniej wartości:\\
\\
\texttt{The current repository location is: -\\
Please enter new repository location and press ENTER.\\
If you want to cancel repository change, please type "cancel" as a value and press ENTER.}\\
\\
W przypadku chęci wycofania się ze zmiany lokalizacji repozytorium należy wpisać \textbf{cancel} i zatwierdzić. W przypadku podania wartości komunikat o braku zdefiniowanego miejsca repozytorium po pierwszym uruchomieniu programu zniknie, konkretnie te pole:\\
\texttt{WARNING: repository location is not set}\\
\\
Druga opcja w menu głównym, \textbf{(2) Show all files in repository} pozwala na przejrzenie wszystkich plików znajdujących się w repozytorium. W przypadku posiadania pustego repozytorium użytkownik zostanie poinformowany o tym fakcie:\\
\\
\texttt{The list is empty. There is nothing to watch.\\
Press any key to back to main menu.}\\
\\
W innym wypadku zostanie użytkownikowi wyświetlona zawartość repozytorium:\\
\texttt{This is an actual list of files in the repository:\\
System file name: windows.iso\\
Description: Instalator Windowsa XP Home Edition SP3\\
User defined file name: windows\\
Location: NAS\\
\\
\\
Press any key to back to main menu.}\\
\\
Opcja \textbf{(3) Find specified Repo file} pozwala na znalezienie istniejącego elementu na liście poprzez jedną z wybranych opcji:\\
\\
\texttt{How do you want to find files?}\\
\texttt{|}\\
\texttt{|---(1) Through system file name} - wyszukuje poprzez nazwę pliku w systemie\\
\texttt{|---(2) Through description} - wyszukuje poprzez opis pliku w systemie\\
\texttt{|---(3) Through user defined file name} - wyszukuje poprzez zdefiniowaną przez użytkownika nazwę pliku\\
\texttt{|---(4) Through file location} - wyszukuje poprzez lokalizację pliku\\
\texttt{|---(5) Go back to the main menu} - wraca do menu głównego\\
\\
Wybierając jedną z interesujących kategorii jesteśmy pytani o podanie wybranej frazy wyszukiwania:\\
\texttt{Please enter specified file system name to search:}\\
Po zdefiniowaniu frazy w przypadku znalezienia wyniku pojawi się wynik jak na przykładzie:\\
\\
\texttt{The results are:\\
System file name: windows.iso\\
Description: Instalator Windowsa XP Home Edition SP3\\
User defined file name: windows\\
Location: NAS\\
\\
\\
Press any key to back to main menu.}\\
\\
W przypadku braku wyników dla wpisanej frazy nie zostanie wyświetlony żaden wynik i użytkownik zostanie poproszony o powrót do poprzedniego pola w menu. W przypadku braku plików do repozytorium użytkownik zostanie o tym poinformowany przed zdefiniowaniem parametru wyszukwania:\\
\\
\texttt{The actual list is empty. There is nothing to search.\\
\\
Press any key to continue.}\\
\\
Czwarta opcja, \textbf{(4) Add Repo file} pozwala na dodanie nowego pliku do repozytorium. Po wejściu w tę opcję użytkownik jest kolejno proszony o podane nazwę pliku w systemie, jej krótki opis, własną zdefiniowaną nazwę pliku oraz jego lokację. Po wypełnieniu pól pliku użytkownik widzi tego wynik, a nowy plik jest zapisywany w repozytorium:\\
\\
\texttt{You have added a new file with following data:\\
System file name: windows\\
Description: Instalator Windowsa XP Home Edition SP3\\
User defined file name: windows xp.iso\\
Location: NAS\\
\\
\\
Press anything to continue.}\\
\\
\\
\textbf{(5) Delete Repo file} usuwa istniejący plik w repozytorium po wybraniu jednego z kryteriów wyszukiwania pliku, który ma być docelowo usunięty. W przypadku braku plików w repozytorium użytkownik otrzyma o tym komunikat:\\
\\
\texttt{The actual list is empty. There is nothing to delete.\\
\\
Press any key to continue.}\\
\\
W przypadku istnienia plików w repozytorium wybór opcji parametrów jest następujący:\\
\\
\texttt{How do you want to find files which are going to be delete?}\\
\texttt{|}\\
\texttt{|---(1) Through system file name} - poprzez wyszukanie nazwy pliku w systemie\\
\texttt{|---(2) Through description} - poprzez wyszkukanie opisu pliku w systemie\\
\texttt{|---(3) Through user defined file name} - poprzez wyszukanie zdefiniowanej przez użytkownika nazwy pliku\\
\texttt{|---(4) Through file location} - poprzez wyszukanie lokalizacji pliku\\
\texttt{|---(5) Forget about searching, just show me all files} - poprzez wybranie pliku z listy wszystkich plików w repozytorium\\
\texttt{|---(6) Go back to the main menu} - opcja pozwala na powrót do menu głównego\\
\\
W przypadku podopcji 1-4 po zdefiniowaniu parametru wyszukiwania program wyświetla dostępne pliki wraz z ich indeksami, a następnie oczekuje na wybranie przez użytkownika jednego z nich poprzez podanie wartości indeksu:
\\
\\
\texttt{The actual file list is:
System file name: windows xp.iso\\
Description: Instalator Windowsa XP Home Edition SP3\\
User defined file name: Windows\\
Location: NAS\\
\\
\\
\\
Please enter desired object index to be deleted: 1\\
Done, file deleted.}\\
\\
Opcja \textbf{(6) Edit Repo file} pozwala na edycję istniejących plików w repozytorium. Podobnie jak w przypadku dodawania i usuwania, jeśli repozytorium nie posiada żadnych plików użytkownik zostanie o tym poinformowany z brakiem możliwości wykorzystania dalszych opcji dopóki nie pojawi(ą) się w repozytorium plik(i). Poniżej wspomniany komunikat:\\
\\
\\texttt{The actual list is empty. There is nothing to edit.\\
\\
Please enter any key to continue.}\\
\\
W przypadku istnienia plików w repozytorium pojawi się lista wszystkich obiektów wraz z ich indeksami, a następnie użytkownik zostanie poproszony o wybranie indeksu w celu edycji.\\
\\
\texttt{The actual file list is:
System file name: windows xp.iso\\
Description: Instalator Windowsa XP Home Edition SP3\\
User defined file name: Windows\\
Location: NAS\\
\\
Object index: 2\\
System file name: linux.iso\\
Description: Instalator Ubuntu 18.04 Desktop amd64\\
User defined file name: Linux\\
Location: NAS\\
\\
Please enter desired object index to be edited: 1\\
}\\
Następnie należy wybrać właściwość pliku do edycji przy czym 1 to systemowa nazwa pliku, 2 to opis pliku, 3 to zdefiniowana przez użytkownika nazwa pliku, a 4 to lokalizacja pliku. Po wyborze należy wprowadzić zmienianą wartość.\\
\\
\texttt{Select edited value type: (1) System filename, (2) File description, (3) User defined filename, (4) file location\\
1} - zdefiniowano zmianę systemowej nazwy pliku\\
\texttt{Please enter the value you want to edit: nowawartosc\\
Done, file edited.\\
\\
Please enter any key to continue.}\\
\\
Po wykonaniu czynności użytkownik może wrócić do menu.\\
\\
Opcja \textbf{(7) Save changes into repository} pozwala na zapis aktualnego stanu repozytorium do pliku. Lokalizacja pliku zawierającego bazę jest stała dla każdego uzytkownika systemu Windows: \%appdata\%\textbackslash FileFinder.dat, np dla uzytkownika serba.r to C:\textbackslash Users\textbackslash serba.r\textbackslash AppData\textbackslash Roaming.\\
W przypadku jakiejkolwiek modyfikacji repozytorium (dodanie i usuwanie plików, edycja istniejących plików) w menu głównym programu pojawi się komunikat o niezapisanych zmianach:\\

\texttt{WARNING: there are unsaved changes in the repository}\\
\\
Komunikat nie jest wyświetlany po zapisaniu zmian. Po uruchomieniu funkcji wykonywany jest zapis i po zakończeniu użytkownik otrzymuje komunikat (:\\
\\
Saved successfully!\\
Press any key to go back to the main menu.\\
\\
Opcja \textbf{(8) Help} wyświetla informacje na temat pomocy, która de facto kieruje do tego sprawozdania, opcja \textbf{(9) About the project} wyświetla informacje na temat projektu w języku angielskim jak i treść zadania oraz opcja \textbf{(0) Exit} kończy pracę programu.

\section{Specyfikacja wewnętrzna}
\textbf{Klasa Repository}:\\
Klasa służy do przechowywania, zapisu oraz odczytu z pliku informacji dotyczących lokalizacji repozytorium.\\

\textit{Repository();}\\
Domyślny konstruktor klasy. Przypisuje wartość "-" do RepoLocation oraz uruchamia metodę RepoStartup().\\

\textit{int FirstRun();}\\
Metoda wywoływana przez RepoStartup() w przypadku braku pliku konfiguracyjnego repozytorium w katalogu użytkownika (\%appdata\%\textbackslash FileFinder.cfg). Genereruje ona taki plik oraz zapisuje jej wartość domyślną RepoLocation.\\

\textit{void SetRepoLocation(std::wstring NewLocationAddress);}\\
Ustawia nową wartość RepoLocation dla obiektu tej klasy.\\

\textit{std::wstring GetRepoLocation();}\\
Wynikiem metody jest wartość RepoLocation dla obiektu tej klasy.\\

\textit{void RepoStartup();}\\
Metoda sprawdza, czy plik konfiguracyjny istnieje. Jeśli tak - wczytuje wartość lokalizacji repozytorium, jeśli nie - uruchamia FirstRun().\\

\textit{std::wstring RepoLocation;}\\
Pole definiujące miejsce repozytorium w systemie.\\
\\\\
\textbf{Klasa Menu}:\\
Klasa ma za zadanie wygenerowanie użytkownikowi uproszczonego interfejsu (z wykorzystaniem prostych pętl) dzięki ktoremu będzie w stanie wykorzystać wszystkie założone funkcjonalności programu.\\

\textit{Menu(Repository& startRepo, RepoManager& startManager);}\\
Konstruktor przyjmuje na wejściu referencję na obiekt repozytorium, aby można było względem niego wykonywać operacje. Podobnie wygląda w przypadku referencji na obiekt klasy RepoManager.\\

\textit{static void clearConsole();}\\
Metoda statyczna służąca do wyczyszczenia konsoli w celu wygenerowania podmenu lub wyświetlenia na nowo konkretnej funkcjonalności w przyjazny dla oka sposób. Metoda zastąpiła początkowo wykorzystywanie odwołanie systemowe \textbf{cls}, co jest ogólnie przyjętą złą praktyką w budowaniu aplikacji konsolowych.\\

\textit{void ChangeRepoLocation();}\\
Metoda zmienia aktualną wartość repozytorium z poziomu menu.\\

\textit{void IsRepositorySet();}\\
Metoda sprawdza czy lokalizacja repozytorium została zmieniona po pierwszym uruchomieniu. Jeśli nie, wyświetla w menu głównym ostrzeżenie.\\

\textit{void IsChangesUnsaved();}\\
Metoda sprawdza czy dokonano zmian na liście obiektu RepoManager. Jeśli tak, wyświetla w menu głównym ostrzeżenie.\\

\textit{void GenerateMainOptions();}\\
Wyświetla elementy menu głównego.\\

\textit{void GenerateSearchOptions();}\\
Wyświetla elementy podmenu dla wyszukiwania plików.\\

\textit{void GenerateDeleteOptions();}\\
Wyświetla elementy podmenu dla usuwania plików.\\

\textit{void OpenMain();}\\
Metoda otwiera menu główne (wywołuje metodę GenerateMainOptions(), IsRepositorySet(), IsChangeUnsaved()) i obsługuje wejścia do do podmenu.\\

\textit{void ShowAllFiles();}\\
Metoda wywołująca listę wszystkich aktualnie obiektów znajdujących się na liście RepoManager jeśli nie jest pusta.\\

\textit{void FindRepoFiles();}\\
Otwiera submenu służące do wyboru parametru wyszukiwania, wywołuje GenerateSearchOption(). W przypadku, gdy lista jest pusta wyświetla komunikat o tym i pozwala na powrót do menu głównego\\

\textit{void FindThroughSystemName();}\\
Metoda prosi o podanie frazy wyszukiwania, a następnie przeszukuje listę sprawdzając czy fraza wyszukiwania jest taka sama jak pole fileSystemName. Jeśli tak - cały obiekt z pasującą wartością jest wyświetlany.\\

\textit{void FindThroughUserDefinedName();}\\
Metoda prosi o podanie frazy wyszukiwania, a następnie przeszukuje listę sprawdzając czy fraza wyszukiwania jest taka sama jak pole fileUserDefinedName. Jeśli tak - cały obiekt z pasującą wartością jest wyświetlany.\\

\textit{void FindThroughDescription();}\\
Metoda prosi o podanie frazy wyszukiwania, a następnie przeszukuje listę sprawdzając czy fraza wyszukiwania jest taka sama jak pole fileDescription. Jeśli tak - cały obiekt z pasującą wartością jest wyświetlany.\\

\textit{void FindThroughLocation();}\\
Metoda prosi o podanie frazy wyszukiwania, a następnie przeszukuje listę sprawdzając czy fraza wyszukiwania jest taka sama jak pole fileLocation. Jeśli tak - cały obiekt z pasującą wartością jest wyświetlany.\\

\textit{void AddRepoFile();}\\
Metoda dodaje nowy obiekt RepoFile do listy RepoManager przedtem pobierając dane do ich pól od użytkownika.

\textit{void DeleteRepoFile();}\\
Otwiera submenu służące do wyboru parametru wyszukiwania do usunięcia pliku, wywołuje GenerateDeleteOption(). W przypadku, gdy lista jest pusta wyświetla komunikat o tym i pozwala na powrót do menu głównego\\

\textit{void SaveChanges();}\\
Zapisuje całą listę typu RepoManager do pliku \%appdata\%\textbackslash FileFinder.dat. Poprawne zapisanie pliku wyłącza komunikat o niezapisanych zmianach w menu głównym.

\textit{void Help();}\\
Wyświetla krótką informację, aby jeśli użytkownik chciał się dowiedzieć więcej o obsłudze programu to niech by zajrzał do tego sprawozdania.\\

\textit{void About();}\\
Wyświetla treść tego zadania i kilka słów od siebie w języku angielskim.\\

\textit{void DeleteThroughSystemName();}\\
Metoda prosi o podanie frazy wyszukiwania, a następnie przeszukuje listę sprawdzając czy fraza wyszukiwania jest taka sama jak pole fileSystemName. Jeśli tak - cały obiekt z pasującą wartością jest wyświetlany wraz z przypisanym do obiektu indeksem. Następnie użytkownik wskazuje jeden z indeksów. Po wskazaniu dostępnej wartości metoda przesuwa iterator o wartość z indeksu-1 i usuwa obiekt z listy.\\

\textit{void DeleteThroughDescription();}\\
Metoda prosi o podanie frazy wyszukiwania, a następnie przeszukuje listę sprawdzając czy fraza wyszukiwania jest taka sama jak pole fileDescription. Jeśli tak - cały obiekt z pasującą wartością jest wyświetlany wraz z przypisanym do obiektu indeksem. Następnie użytkownik wskazuje jeden z indeksów. Po wskazaniu dostępnej wartości metoda przesuwa iterator o wartość z indeksu-1 i usuwa obiekt z listy.\\

\textit{void DeleteThroughUserDefinedName();}\\
Metoda prosi o podanie frazy wyszukiwania, a następnie przeszukuje listę sprawdzając czy fraza wyszukiwania jest taka sama jak pole fileUserDefinedName. Jeśli tak - cały obiekt z pasującą wartością jest wyświetlany wraz z przypisanym do obiektu indeksem. Następnie użytkownik wskazuje jeden z indeksów. Po wskazaniu dostępnej wartości metoda przesuwa iterator o wartość z indeksu-1 i usuwa obiekt z listy.\\

\textit{void DeleteThroughLocation();}\\
Metoda prosi o podanie frazy wyszukiwania, a następnie przeszukuje listę sprawdzając czy fraza wyszukiwania jest taka sama jak pole fileLocation. Jeśli tak - cały obiekt z pasującą wartością jest wyświetlany wraz z przypisanym do obiektu indeksem. Następnie użytkownik wskazuje jeden z indeksów. Po wskazaniu dostępnej wartości metoda przesuwa iterator o wartość z indeksu-1 i usuwa obiekt z listy.\\

\textit{void ShowAllFilesToDelete();}\\
Metoda w stosunku do poprzednich wyświetla listę wszystkich obiektów wraz z indeksami. Następnie użytkownik wskazuje jeden z indeksów. Po wskazaniu dostępnej wartości metoda przesuwa iterator o wartość z indeksu-1 i usuwa obiekt z listy.\\

\textit{void EditRepoFile();}\\
Metoda w przypadku niepustej listy wywołuje metodę EditFileFromWholeList() dla obiektu RepoManager mającej na celu faktyczną edycję wybranego obiektu.\\

\textit{Repository& actualRepo;}\\
Pole przechowujące referencję na wykorzystywany obiekt repozytorium.\\

\textit{RepoManager& actualManager;}\\
Pole przechowujące referencję na wykorzystywany obiekt RepoManager przechowujący listę.\\
\\\\
\textbf{Klasa RepoFile}:\\
Klasa definiująca pliki znajdujące się w repozytorium.\\

\textit{RepoFile();}\\
Domyślny konstruktor klasy.

\textit{RepoFile(std::wstring inputSystemName, std::wstring inputUserDefinedName, std::wstring inputFileLocation, std::wstring inputFileDesc);}\\
Konstruktor pozwalający na zdefiniowanie wszystkich pól klasy.\\


\textit{virtual ~RepoFile();}\\
Destruktor klasy.\\

\textit{void SetSystemName(std::wstring newFileName);}\\
Ustawia wartość fileSystemName poprzez podany parametr.\\

\textit{std::wstring GetSystemName();}\\
Zwraca aktualną wartość pola prywatnego fileSystemName.\\

\textit{void SetUserDefinedName(std::wstring newFileFileName);}\\
Ustawia wartość fileUserDefinedName poprzez podany parametr.\\

\textit{std::wstring GetUserDefinedName();}\\
Zwraca aktualną wartość pola prywatnego fileUserDefinedName.\\

\textit{void SetFileDesc(std::wstring newFiledesc);}\\
Ustawia wartość fileDescription poprzez podany parametr.\\

\textit{std::wstring GetFileDesc();}\\
Zwraca aktualną wartość pola prywatnego fileDescription.\\

\textit{void SetFileLocation(std::wstring newFilelocation);}\\
Ustawia wartość fileLocation poprzez podany parametr.\\

\textit{std::wstring GetFileLocation();}\\
Zwraca aktualną wartość pola prywatnego fileLocation.\\

\textit{RepoFile& operator=(RepoFile &right);}\\
Operator przypisania dla klasy RepoFile.\\

\textit{friend std::wostream& operator<<(std::wostream& output, RepoFile& right);}\\
Operator wypisania dla klasy RepoFile, wykorzystywany tylko do wyświetlania zawartości pól obiektu.\\

\textit{friend std::wistream& operator>>(std::wistream& input, RepoFile& right);}\\
Operator wpisania dla klasy RepoFile, docelowo niewykorzystywany.\\

\textit{std::wstring fileSystemName;}\\
Pole definiuje nazwę pliku w systemie.\\

\textit{std::wstring fileUserDefinedName;}\\
Pole definiuje nazwę zdefiniowaną przez użytkownika dla pliku.\\

\textit{std::wstring fileDescription;}\\
Pole definiuje opis pliku.\\

\textit{std::wstring fileLocation;}\\
Pole definiuje lokalizację pliku.\\
\\\\
\textbf{Klasa RepoManager}:\\
Klasa ma na celu dostarczenie listy przechowującej oraz dostarczenie jej pełnej obsługi (takie jak edytowanie, dodawanie, usuwanie i wyszukiwanie elementów).\\

\textit{RepoManager();}\\
Konstruktor klasy. W praktyce wykorzystany raz, w main().\\

\textit{virtual ~RepoManager();}\\
Destruktor klasy.\\

\textit{void AddToRepo(RepoFile& inputRepoFile);}\\
Dodaje obiekt na koniec listy oraz zaznacza zmianę w repozytorium.\\

\textit{void ShowAllFiles();}\\
Wyświetla wszystkie obiekty na liście korzystając z operatora wypisania dla obiektu RepoFile.\\

\textit{bool IsRepoEmpty()}\\
Zwraca wartość true w przypadku, gdy lista nie jest pusta. W przeciwnym wypadku zwraca false.

\textit{void FindInRepoBySystemName(std::wstring inputName);}\\
Metoda iteruje po całej liście porównując pole fileSystemName z inputName i w przypadku tych samych wartości wyświetla cały obiekt.\\

\textit{void FindInRepoByDesc(std::wstring inputDesc);}\\
Metoda iteruje po całej liście porównując pole fileDescription z inputDesc i w przypadku tych samych wartości wyświetla cały obiekt.\\

\textit{void FindInRepoByUserDefinedName(std::wstring inputFileName);}\\
Metoda iteruje po całej liście porównując pole fileUserDefinedName z inputFileName i w przypadku tych samych wartości wyświetla cały obiekt.\\

\textit{void FindInRepoByLocation(std::wstring inputLocation);}\\
Metoda iteruje po całej liście porównując pole fileLocation z inputLocation i w przypadku tych samych wartości wyświetla cały obiekt.\\

\textit{void RepoLoadFromFile();}\\
Metoda sprawdza, czy plik w lokalizacji \%appdata\%\textbackslash FileFinder.dat istnieje i jeśli tak, wczytuje zawartość do listy. Metoda jest wywoływana w trakcie startu programu.\\

\textit{bool RepoSaveToFile();}\\
Metoda sprawdza, czy plik w lokalizacji \%appdata\%\textbackslash FileFinder.dat istnieje i jeśli tak, zapisuje zawartość list do pliku.\\

\textit{void DeleteFromRepoBySystemName(std::wstring inputName);}\\
Metoda iteruje po całej liście porównując pole fileSystemName z inputName i w przypadku tych samych wartości wyświetla cały obiekt z indeksem ze zmiennej iter oraz zapisaniem wartości iter do wektora listIter. Następnie użytkownik ma za zadanie wskazanie jednego z obiektów do usunięcia, a ten jest sprawdzany czy jest rzeczywiscie wartością int oraz czy podana wartość int znajduje się we wspomnianym wektorze. Po spełnieniu warunku advance przesuwa iterator o wartość indeksu-1 oraz usuwa wyświetlony obiekt z listy.\\

\textit{void DeleteFromRepoByDesc(std::wstring inputDesc);}\\
Metoda iteruje po całej liście porównując pole fileDescription z inputDesc i w przypadku tych samych wartości wyświetla cały obiekt z indeksem ze zmiennej iter oraz zapisaniem wartości iter do wektora listIter. Następnie użytkownik ma za zadanie wskazanie jednego z obiektów do usunięcia, a ten jest sprawdzany czy jest rzeczywiscie wartością int oraz czy podana wartość int znajduje się we wspomnianym wektorze. Po spełnieniu warunku advance przesuwa iterator o wartość indeksu-1 oraz usuwa wyświetlony obiekt z listy.\\

\textit{void DeleteFromRepoByUserDefinedName(std::wstring inputFileName);}\\
Metoda iteruje po całej liście porównując pole fileUserDefinedName z inputFileName i w przypadku tych samych wartości wyświetla cały obiekt z indeksem ze zmiennej iter oraz zapisaniem wartości iter do wektora listIter. Następnie użytkownik ma za zadanie wskazanie jednego z obiektów do usunięcia, a ten jest sprawdzany czy jest rzeczywiscie wartością int oraz czy podana wartość int znajduje się we wspomnianym wektorze. Po spełnieniu warunku advance przesuwa iterator o wartość indeksu-1 oraz usuwa wyświetlony obiekt z listy.\\

\textit{void DeleteFromRepoByLocation(std::wstring inputLocation);}\\
Metoda iteruje po całej liście porównując pole fileLocation z inputLocation i w przypadku tych samych wartości wyświetla cały obiekt z indeksem ze zmiennej iter oraz zapisaniem wartości iter do wektora listIter. Następnie użytkownik ma za zadanie wskazanie jednego z obiektów do usunięcia, a ten jest sprawdzany czy jest rzeczywiscie wartością int oraz czy podana wartość int znajduje się we wspomnianym wektorze. Po spełnieniu warunku advance przesuwa iterator o wartość indeksu-1 oraz usuwa wyświetlony obiekt z listy.\\

\textit{void DeleteFromRepoWholeList();}\\
Metoda iteruje po całej liście wyświetlając obiekty z indeksem ze zmiennej iter oraz zapisaniem wartości iter do wektora listIter. Następnie użytkownik ma za zadanie wskazanie jednego z obiektów do usunięcia, a ten jest sprawdzany czy jest rzeczywiscie wartością int oraz czy podana wartość int znajduje się we wspomnianym wektorze. Po spełnieniu warunku advance przesuwa iterator o wartość indeksu-1 oraz usuwa wyświetlony obiekt z listy.\\

\textit{void EditFileFromWholeList();}\\
Metoda iteruje po całej liście wyświetlając obiekty z indeksem ze zmiennej iter oraz zapisaniem wartości iter do wektora listIter. Następnie użytkownik ma za zadanie wskazanie jednego z obiektów do edycji, a ten jest sprawdzany czy jest rzeczywiscie wartością int oraz czy podana wartość int znajduje się we wspomnianym wektorze. Po spełnieniu warunku advance przesuwa iterator o wartość indeksu-1 oraz wysyła iterator w parametrze do metody ChangeSelectedValueInsideList(). Po jej zakończeniu usuwa wskazany obiekt.\\

\textit{void ChangeSelectedValueInsideList(std::list<RepoFile>::iterator inputIterator);}\\
Metoda tworzy kopię obiektu na który wskazuje iterator, a następnie użytkownik ma za zadanie wskazanie pola obiektu RepoFile do edycji za pomocą wartości 1-4 w  zmiennej selectedValue, a ta jest sprawdzana czy jest rzeczywiscie wartością int oraz czy podana wartość int znajduje się w wektorze availableValues składającego się z wartości 1-4, przy czym 1 oznacza edycję pola fileSystemName, 2 oznacza edycję pola fileDescription, 3 oznacza edycję pola fileUserDefinedName i 4 oznacza edycję pola fileLocation. Edytowana jest kopia obiektu, a następnie po edycji kopia jest dodawana do listy.\\

\textit{std::list<RepoFile> repoList;}\\
Pole listy przechowującej obiekty RepoFile będącymi plikami repozytorium.\\

\textit{std::wstring listPath;}\\
Pole przechowujące lokalizację pliku przechowującego listę, konkretnie \%appdata\%\textbackslash FileFinder.dat.\\

\\\\
\textbf{Klasa ValueChecker}:\\
Klasa pomocnicza mająca na celu drobne poprawki w błędach działania programu.\\

\textit{static void IfInt();}\\
Metoda statyczna sprawdzająca czy zczytywana wartość do zmiennej int jest rzeczywiście tego typu. Funkcja szczególnie przydatna w pętlach.

\textit{static void DeleteLastCharacter(std::wstring filePath);}\\
Metoda statyczna mająca na celu usunięcie ostatniego znaku z pliku.


\section{Testowanie}

Test 1:\\
Program został uruchomiony w następujący sposób: FileFinder.exe\\
Wynik działania programu: ponieważ to było pierwsze uruchomienie programu, ten wyświetlił informację o utworzeniu pliku konfiguracyjnego dla repozytorium.\\
\texttt{This is the first run of FileFinder. The configuration file will be created in:\\
C:\textbackslash Users\textbackslash serba.r\textbackslash AppData\textbackslash Roaming\textbackslash FileFinder.cfg}\\
\\
Test 2:\\
Program został uruchomiony po raz drugi w następujący sposób: FileFinder.exe\\
Wynik działania programu: ponieważ to nie było pierwsze uruchomienie programu, ten nie wyświetlił informacji o utworzeniu pliku konfiguracyjnego dla repozytorium, lecz od razu przeszedł do menu głównego programu:\\
\texttt{WARNING: repository location is not set\\
\\
Welcome in Repofinder! Choose desired operation by typing a number:\\
|\\
|---(1) Set new repository location\\
|---(2) Show all files in repository\\
|---(3) Find specified Repo file\\
|---(4) Add Repo file\\
|---(5) Delete Repo file\\
|---(6) Edit Repo file\\
|---(7) Save changes into repository\\
|---(8) Help\\
|---(9) About the project\\
|---(0) Exit}\\
\\
\end{document}
