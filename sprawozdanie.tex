
\documentclass[10pt, a4paper]{article}


% % % Ustawienia strony
\usepackage[margin=0.5in]{geometry}

% % % Język polski
\usepackage[utf8]{inputenc}                                      
\usepackage[T1]{fontenc}                            
\usepackage[polish]{babel}  

% % % Środowisko 
\usepackage{verbatim}


% % % Nagłówek pliku
\newcommand{\KNDate}[1]{#1, \today}
\newcommand{\KNPrzedmiot}[1]{\textbf{#1}}
\newcommand{\KNTemat}[1]{Temat: #1}
\newcommand{\KNSemestr}[1]{Semestr: #1}
\newcommand{\KNGrupa}[1]{Grupa: #1}
\newcommand{\KNAutor}[1]{Autor: #1}
\newcommand{\KNProwadzacy}[1]{Prowadzący: #1}

\newcommand{\KNNaglowek}[8]
{
	\newpage
	\begin{center}
		{
			% % % Brak numeracji strony
			\thispagestyle{empty}	
			\raggedleft \KNDate{#1}
					
			\vspace{5cm}	
			\centering
			\Huge \KNPrzedmiot{#2}\\
			\vspace{1cm}	
			\LARGE \KNTemat{#3} \\
			
			\vspace{15cm}
			\normalsize
			\raggedleft
			\KNAutor{#5}\\
			\KNSemestr{#6}\\
			\KNGrupa{#7}\\
			\vspace{1cm}
			\KNProwadzacy{#4}\\

		
			\newpage	
		} 
	\end{center}
}


\begin{document}
	
\KNNaglowek{Gliwice}
           {Programowanie Komputerów}
           {Baza płyt z oprogramowaniem instalacyjnym}
           {dr inż. Karolina Nurzyńska}
           {Radosław Serba}
           {trzeci}
           {II}
           

\section{Treść zadania} 
Zadaniem programu jest zarządzanie zbiorem plików z oprogramowaniem instalacyjnym. Zarządzanie obejmuje: przeszukiwanie zasobów, dodawanie nowych pozycji, modyfikacje oraz usuwanie istniejących rekordów.
\section{Analiza rozwiązania}
Głównym zadaniem programu jest przechowywanie informacji o przechowywanych plikach oprogramowania instalacyjnego w postaci wide stringów. Przy pierwszym otwarciu programu ten poinformuje nas o tym oraz o utworzeniu pliku konfiguracyjnego przechowującego repozytorium.\\

Następnie program wyświetli użytkownikowi wszystkie możliwe opcje do wykonania począwszy od zmiany lokalizacji repozytorium po dodawanie, usuwanie, edytowanie i wyszukiwanie plików wedle wskazanych kryteriów. Użytkownik porusza się w programie za pomocą uproszczonego menu głównego, przy czym aby się po menu poruszać należy naciskać przyciski cyfr odpowiadjących funkcjom, które użytkownik chce wykorzystać.\\

Do wykonania programu zostały wykorzystane klasy \textbf{Repository} do obsługi zapisu pliku konfiguracyjnego przechowującego repozytorium, \textbf{Menu} do wyświetlania interfejsu użytkownika oraz do zapewnienia dostępu do wszystkich funkcjonalności programu, \textbf{RepoFile} do definicji plików instalacyjnych w postaci obiektów, \textbf{RepoManager} do obsługi listy wspomnianych plików instalacyjnych wraz ze wszystkimi operacjami wspomnianymi w treści zadania oraz \textbf{ValueChecker} jako klasa uzupełniająca do uniknięcia błędów wynikających z naturalnego działania niektórych funkcji w języku C++ dla wcześniej wspominanych klas.
\section{Specyfikacja zewnętrzna}
Program należy uruchomić poprzez kliknięcie dwukrotnie lewym przyciskiem myszy na ikonę w Eksploratorze Windows lub poprzez linię poleceń (cmd.exe) bez podania argumentów, w następujący sposób:\\
\texttt{<FileFinder.exe>}\\
\\
Po uruchomieniu programu, w przypadku jego pierwszego uruchomienia wyświetli komunikat:\\
\texttt{This is the first run of FileFinder. The configuration file will be created in:}\\
\texttt{C:\textbackslash Users\textbackslash <nazwa użytkownika>\textbackslash AppData\textbackslash Roaming\textbackslash FileFinder.cfg}\\
\\
Następnie po naciśnięciu dowolnego klawisza zostaniemy przekierowani do menu głównego programu:\\
\texttt{WARNING: repository location is not set\\
\\
Welcome in Repofinder! Choose desired operation by typing a number:\\
|\\
|---(1) Set new repository location\\
|---(2) Show all files in repository\\
|---(3) Find specified Repo file\\
|---(4) Add Repo file\\
|---(5) Delete Repo file\\
|---(6) Edit Repo file\\
|---(7) Save changes into repository\\
|---(8) Help\\
|---(9) About the project\\
|---(0) Exit}\\
\\
Wybierając opcję (1) jesteśmy przenoszeni do podmenu pozwalającego na ustawienie nowego miejsca na repozytorium plików poprzez podanie odpowiedniej wartości:\\
\\
\texttt{The current repository location is: -\\
Please enter new repository location and press ENTER.\\
If you want to cancel repository change, please type "cancel" as a value and press ENTER.}\\
\\
W przypadku chęci wycofania się ze zmiany lokalizacji repozytorium należy wpisać \textbf{cancel} i zatwierdzić. W przypadku podania wartości komunikat o braku zdefiniowanego miejsca repozytorium po pierwszym uruchomieniu programu zniknie, konkretnie te pole:\\
\texttt{WARNING: repository location is not set}\\
\\
Druga opcja w menu głównym, \textbf{(2) Show all files in repository} pozwala na przejrzenie wszystkich plików znajdujących się w repozytorium. W przypadku posiadania pustego repozytorium użytkownik zostanie poinformowany o tym fakcie:\\
\\
\texttt{The list is empty. There is nothing to watch.\\
Press any key to back to main menu.}\\
\\
W innym wypadku zostanie użytkownikowi wyświetlona zawartość repozytorium:\\
\texttt{This is an actual list of files in the repository:\\
System file name: windows.iso\\
Description: Instalator Windowsa XP Home Edition SP3\\
User defined file name: windows\\
Location: NAS\\
\\
\\
Press any key to back to main menu.}\\
\\
Opcja \textbf{(3) Find specified Repo file} pozwala na znalezienie istniejącego elementu na liście poprzez jedną z wybranych opcji:\\
\\
\texttt{How do you want to find files?}\\
\\
\texttt{|---(1) Through system file name} - wyszukuje poprzez nazwę pliku w systemie\\
\texttt{|---(2) Through description} - wyszukuje poprzez opis pliku w systemie\\
\texttt{|---(3) Through user defined file name} - wyszukuje poprzez zdefiniowaną przez użytkownika nazwę pliku\\
\texttt{|---(4) Through file location} - wyszukuje poprzez lokalizację pliku\\
\texttt{|---(5) Go back to the main menu} - wraca do menu głównego\\
\\
Wybierając jedną z interesujących kategorii jesteśmy pytani o podanie wybranej frazy wyszukiwania:\\
\texttt{Please enter specified file system name to search:}\\
Po zdefiniowaniu frazy w przypadku znalezienia wyniku pojawi się wynik jak na przykładzie:\\
\\
\texttt{The results are:\\
System file name: windows.iso\\
Description: Instalator Windowsa XP Home Edition SP3\\
User defined file name: windows\\
Location: NAS\\
\\
\\
Press any key to back to main menu.}\\
\\
W przypadku braku wyników dla wpisanej frazy nie zostanie wyświetlony żaden wynik i użytkownik zostanie poproszony o powrót do poprzedniego pola w menu. W przypadku braku plików do repozytorium użytkownik zostanie o tym poinformowany przed zdefiniowaniem parametru wyszukwania:\\
\\
\texttt{The actual list is empty. There is nothing to search.\\
\\
Press any key to continue.}\\
\\
Czwarta opcja, \textbf{(4) Add Repo file} pozwala na dodanie nowego pliku do repozytorium. Po wejściu w tę opcję użytkownik jest kolejno proszony o podane nazwę pliku w systemie, jej krótki opis, własną zdefiniowaną nazwę pliku oraz jego lokację. Po wypełnieniu pól pliku użytkownik widzi tego wynik, a nowy plik jest zapisywany w repozytorium:\\
\\
\texttt{You have added a new file with following data:\\
System file name: windows\\
Description: Instalator Windowsa XP Home Edition SP3\\
User defined file name: windows xp.iso\\
Location: NAS\\
\\
\\
Press anything to continue.}\\
\\
\textbf{(5) Delete Repo file} usuwa istniejący plik w repozytorium po wybraniu jednego z kryteriów wyszukiwania pliku, który ma być docelowo usunięty.
\section{Specyfikacja wewnętrzna}
\textbf{Klasa Repository}:\\

\textit{Repository();}\\
opis


\textit{int firstRun();}\\
opis

\textit{void SetRepoLocation(std::wstring NewLocationAddress);}\\
opis

\textit{std::wstring GetRepoLocation();}\\
opis

\textit{void RepoStartup();}\\
opis

\textit{std::wstring RepoLocation;}\\
opis
\\\\
\textbf{Klasa Menu}:\\

\textit{Menu(Repository& startRepo, RepoManager& startManager);}\\
opis

\textit{static void clearConsole();}\\
opis

\textit{void ChangeRepoLocation();}\\
opis

\textit{void IsRepositorySet();}\\
opis

\textit{void IsChangesUnsaved();}\\
opis

\textit{void GenerateMainOptions();}\\
opis

\textit{void GenerateSearchOptions();}\\
opis

\textit{void GenerateDeleteOptions();}\\
opis

\textit{void OpenMain();}\\
opis

\textit{void ShowAllFiles();}\\
opis

\textit{void FindRepoFiles();}\\
opis

\textit{void FindThroughSystemName();}\\
opis

\textit{void FindThroughUserDefinedName();}\\
opis

\textit{void FindThroughDescription();}\\
opis

\textit{void FindThroughLocation();}\\
opis

\textit{void AddRepoFile();}\\
opis

\textit{void DeleteRepoFile();}\\
opis

\textit{void SaveChanges();}\\
opis

\textit{void Help();}\\
opis

\textit{void About();}\\
opis

\textit{void DeleteThroughSystemName();}\\
opis

\textit{void DeleteThroughDescription();}\\
opis

\textit{void DeleteThroughUserDefinedName();}\\
opis

\textit{void DeleteThroughLocation();}\\
opis

\textit{void ShowAllFilesToDelete();}\\
opis

\textit{void EditRepoFile();}\\
opis

\textit{Repository& actualRepo;}\\
opis

\textit{RepoManager& actualManager;}\\
opis
\\\\
\textbf{Klasa RepoFile}:\\

\textit{RepoFile();}\\
opis

\textit{RepoFile(std::wstring inputSystemName, std::wstring inputUserDefinedName, std::wstring inputFileLocation, std::wstring inputFileDesc);}\\
opis

\textit{RepoFile(std::wstring newFileLocation);}\\
opis

\textit{virtual ~RepoFile();}\\
opis

\textit{void SetSystemName(std::wstring newFileName);}\\
opis

\textit{std::wstring GetSystemName();}\\
opis

\textit{void SetUserDefinedName(std::wstring newFileFileName);}\\
opis

\textit{std::wstring GetUserDefinedName();}\\
opis

\textit{void SetFileDesc(std::wstring newFiledesc);}\\
opis

\textit{std::wstring GetFileDesc();}\\
opis

\textit{void SetFileLocation(std::wstring newFilelocation);}\\
opis

\textit{std::wstring GetFileLocation();}\\
opis

\textit{RepoFile& operator=(RepoFile &right);}\\
opis

\textit{friend std::wostream& operator<<(std::wostream& output, RepoFile& right);}\\
opis

\textit{friend std::wistream& operator>>(std::wistream& input, RepoFile& right);}\\
opis

\textit{std::wstring fileSystemName;}\\
opis

\textit{std::wstring fileUserDefinedName;}\\
opis

\textit{std::wstring fileDescription;}\\
opis

\textit{std::wstring fileLocation;}\\
opis
\\\\
\textbf{Klasa RepoManager}:\\

\textit{RepoManager();}\\
opis

\textit{virtual ~RepoManager();}\\
opis

\textit{void AddToRepo(RepoFile& inputRepoFile);}\\
opis

\textit{void ShowAllFiles();}\\
opis

\textit{void FindInRepoBySystemName(std::wstring inputName);}\\
opis

\textit{void FindInRepoByDesc(std::wstring inputDesc);}\\
opis

\textit{void FindInRepoByUserDefinedName(std::wstring inputFileName);}\\
opis

\textit{void FindInRepoByLocation(std::wstring inputLocation);}\\
opis

\textit{void RepoLoadFromFile();}\\
opis

\textit{bool RepoSaveToFile();}\\
opis

\textit{void DeleteFromRepoBySystemName(std::wstring inputName);}\\
opis

\textit{void DeleteFromRepoByDesc(std::wstring inputDesc);}\\
opis

\textit{void DeleteFromRepoByUserDefinedName(std::wstring inputFileName);}\\
opis

\textit{void DeleteFromRepoByLocation(std::wstring inputLocation);}\\
opis

\textit{void DeleteFromRepoWholeList();}\\
opis

\textit{void EditFileFromWholeList();}\\
opis

\textit{bool changesNotSaved;}\\
opis

\textit{std::list<RepoFile> repoList;}\\
opis

\textit{std::wstring listPath;}\\
opis

\textit{void ChangeSelectedValueInsideList(std::list<RepoFile>::iterator inputIterator);}\\
opis
\\\\
\textbf{Klasa ValueChecker}:\\

\textit{static void IfInt();}\\
Metoda statyczna sprawdzająca czy zczytywana wartość do zmiennej int jest rzeczywiście tego typu. Funkcja szczególnie przydatna w pętlach.

\textit{static void DeleteLastCharacter(std::wstring filePath);}\\
Metoda statyczna mająca na celu usunięcie ostatniego znaku z pliku.


\section{Testowanie}

\end{document}
